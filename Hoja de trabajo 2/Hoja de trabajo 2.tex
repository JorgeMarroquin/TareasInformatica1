\documentclass[10pt,letterpaper]{article}
\usepackage[latin1]{inputenc}
\usepackage[spanish]{babel}
\usepackage{amsmath}
\usepackage{amsfonts}
\usepackage{amssymb}
\usepackage{graphicx}
\usepackage[left=2cm,right=2cm,top=2cm,bottom=2cm]{geometry}

\begin{document}

\begin{center}
        \huge{Hoja de trabajo \#2} \\
\end{center}
\begin{center}
        \textbf{Nombre: }Jorge Armando Marroqu�n Ochoa\\
        \textbf{Carnet: }2018358\\
        \textbf{Correo: }marroquin181358@unis.edu.gt\\
\end{center}
\section{Ejercicio \#1}
\begin{flushleft}
\textbf{Demostrar:}
\end{flushleft}
\[
        \forall\ n.\ n^3\geq n^2
\]
\begin{flushleft}
\textbf{Caso base:}
\begin{enumerate}
\item{$0^3\geq 0^2$}
\item {$0\geq 0$}
\end{enumerate}

\textbf{Caso inductivo:}
\begin{enumerate}
\item{$(n + 1)^3\geq (n + 1)^2$}
\item{$(n + 1)\geq (n + 1)^2 / (n + 1)^2$}
\item{Un n�mero partido por �l mismo es 1: \\$(n + 1)\geq 1$}
\item{$n \geq 1 - 1$}
\item{$n \geq 0$}
\end{enumerate}
\end{flushleft}

\section{Ejercicio \#2}
\begin{flushleft}
\textbf{Demostrar:}
\end{flushleft}
\[
        \forall\ n.\ (1+x)^n\geq nx
\]
\begin{flushleft}
\textbf{Caso base:}
\begin{enumerate}
\item{$(1+0)^0\geq 0(0)$}
\item {Todo n�mero elevado a la 0 es 1:\\ $ 1 \geq 0$}
\end{enumerate}

\textbf{Caso inductivo:}
\begin{enumerate}
\item{$(1+x)^{(n + 1)}\geq (n + 1)x + 1$}
\item{$(1+x)^{(n + 1)}\geq nx + x + 1$}
\item{$(1+x).(1+x)^n\geq nx + x + 1$}

\end{enumerate}
\end{flushleft}
\end{document}
